\section{Философская мысль Беларуси как форма осмысления национальных культурных традиций и становление национального самосознания}
\label{sec:analysis}

В национальной белорусской идее воплощается историческое стремление белорусского народа к свободе, самостоятельности и благосостоянию, сохранению и развитию белорусской науки, белорусского языка и белорусского государства, гуманистических перспектив и гражданской ответственности за будущее страны.
Формирование белорусской идеи как систематизированного обобщения национального самосознания, имеет глубинные корни и представлено, как в рационализированной, социально-философской и общественно-политической форме, так и в образно-типизационном, художественно-литературном выражении. Суть ее заключается в осмыслении бытия белорусского этноса, исторического наследия и борьбы белорусского народа, его национальной идентичности и самости, генетических истоках исторического предназначения, идей сосуществования, основаниях уникальности, особенностях национального характера, геополитического положения и роли в глобализационных процессах современности. Становлению белорусской идеи способствовали ассимиляция духовного опыта западноевропейской и русской традиций в культуре Беларуси, социально-философские и гуманистические идеи в белорусской философии (Ф. Скорина, С. Будный, С. Полоцкий), развитие философско-белорусского самосознания в ХХ в. (А. Гарун, Н. Абдиралович-Канчевский и др.), философско-публицистические выступления и произведения (К. Калиновский, Я. Колас, М. Богданович, Ф. Богушевич, Я. Купала, В. Ластовский).

Мощная волна национального возрождения взламывает старые формы, раскрывает тайны о начале белорусской государственности, исторических событиях и личностях, "белых" пятнах национального самоутверждения и самоидентификации, утверждает величие нашего прошлого и обосновывает веру в будущее белорусского народа, как уникального субъекта истории и органического компонента европейской и мировой цивилизации. Национальная идея является источником духовного обогащения, формирования гуманистического мировоззрения, высоких гражданских качеств, возрождения исторической памяти и национального самосознания, чувства гражданской гордости и патриотизма, национально-культурного возрождения Беларуси.
Национальная идея формируется лишь в контексте традиций национальной культуры, ибо традиция (от лат. — передача, предание) – это способ бытия и воспроизводства элементов социального и культурного наследия, норм поведения, мировоззренческих установок, форм сознания и человеческого общения. Традиция характеризует связь настоящего и прошлого, выступая своего рода посредником между современностью и прошлым, механизмом хранения и передачи образцов, приемов и навыков деятельности. Противоречивость традиции проявляется в том, что она, с одной стороны выглядит как консервация прошлого, символ "отставания", "отсталости" и неизменности, с другой стороны — выступает как необходимое условие сохранения, преемственности и устойчивости человеческого бытия. В то же время развитие культуры невозможно представить без создания новых культурных продуктов и образцов, или так называемой культурной инновации, благодаря чему формируется национальная культура.

Национальная культура — культура определенной нации, сложившаяся на протяжении ее исторического развития на основе этнической культуры. Белорусская национальная культура сложилась на основе культуры белорусского этноса во взаимодействии с культурами других этнических групп — украинцев, русских, литовцев, и др. Своеобразие белорусской культуры определили ее тесные взаимоотношения с другими народами, "пограничный" характер. Белорусская культура на протяжении всего своего развития всегда чувствовала влияние других культур и сама значительно повлияла на соседние культуры. Тесные взаимоотношения были обусловлены географическим положением Беларуси (расположение между Востоком и Западом), прохождением через страну двух больших культурных регионов (влиянием двух миров) — православно-византийского и римско-католического.

В силу особого географического положения Беларуси, находящейся на стыке двух центров — западного римско-католического и восточного православно-византийского, ее территория часто подвергалась постоянному переделу. Возможно, этим объясняется то, что самоидентификация белорусов носила в основном локальный характер и основывалась больше на принадлежности к определенной территории, местности, региону ("тутэйшыя"), социальной группе (православные, католики и т.д.), клану, роду, семье, возвышаясь иногда до уровня нации и государства. Отсутствие непрерывной традиции, воздействие различных культур и цивилизаций затрудняли процессы идентификации белорусов, не давали четких критериев определения культурной принадлежности.

Белорусам сегодня необходимо самоопределиться, "найти самих себя", иметь довольно сил, что гарантировало бы равноправное участие в свободном обмене передовыми идеями и течениями мысли в духовной жизни восточного и западного регионов, помогло процессу самоидентификации и формированию чувства патриотизма. Чувство патриотизма играет особую роль в формировании современных мировоззренческих приоритетов белорусской государственности. Патриотизм (от греч. – patris – родина, отечество) – идея, чувство и действия, выражающие любовь и преданность Родине, способствующие ее успехам во всех сферах внутренней жизни, повышению ее могущества и укреплению авторитета на международной арене. Патриотизм – это осознание общности интересов людей, веками живущих в обособленных отечествах, уважение к историческому прошлому своего народа, гордость за его достижения и горечь за неудачи, беды и ошибки предков и современников, активная деятельность по созданию нового, прогрессивного. Обязательной стороной подлинного патриотизма является уважение к другим народам, их языку, культуре, истории.
Исторический опыт показывает, что государства, достигавшие высоких вершин экономического, политического и культурного развития на определенных исторических этапах, всегда обращались к объединяющим мировоззренческим идеям, выражающим в концентрированном виде цели, к которым стремится общество. Такого рода идеи впитывают в себя духовные ценности, значимые и понятные каждому человеку, вследствие чего они способны выступить в качестве мировоззренческого мобилизующего начала. Так, в свое время немецкая философия, разрабатывая категории абсолютного духа, единства мирового разума, рационального начала в развитии общества и др., выработала систему общенациональных ценностей, которые были реализованы политической практикой создания сильного германского государства из мелких княжеств. Становлению государственности в США также способствовало провозглашение идеологемы "американская мечта", дополненной в период великой депрессии рузвельтовским "новым курсом", а позже идеей "нового общества" вместе с системой долгосрочных программ борьбы с бедностью, расизмом, неграмотностью. Фундаментальная идея "мирового порядка", которая со стороны других государств воспринимается как "мировое государство", характерна для современного американского общества.

Противоречивость русской души, ее амбивалентность во многом объясняются антиномиями исторического пути Руси, которые ей пришлось испытать, ее промежуточным, неустойчивым положением между двумя цивилизациями. Трудолюбие и лень, деспотизм и доброта, бунт и смирение, коллективизм и персонализм, мужское и женское, христианство и язычество, аскетически-монашеское и безбожное, трудолюбие и праздность и т.д. – таковы противоположные начала русского характера. Комплекс раболепия (сервилизма), транслированный от Востока, причудливо сочетался здесь с бунтарским духом, вечным стремлением к свободе.
Духовные ценности белорусов несомненно формировались в контексте восточно-славянского менталитета, традиционно испытывая трудности существования между Востоком и Западом и осуществляя поиск собственного пути развития. Белоруская ментальность впитала в себя и униатскую склонность к компромиссам, и героику католицизма, и строгую воздержанность вместе с индивидуализмом протестантизма. Многие исследователи отмечают, что белорусы миролюбивы, для них нехарактерно чувство национального превосходства над другими национальностями. Говоря о толерантности белорусской нации, обычно выделяют такие черты, как рассудительность и поиск справедливости без насилия, стремление к разумному компромиссу, терпимость, чуткость, уважение людей с иным мировосприятием и стилем мышления.
Система ценностей белорусов формировалась под влиянием западно- и восточнославянской культур. Она имеет много общего с ценностями русского общества (с общеславянскими ценностями). В то же время для нее характерны свои особые, специфические черты. Общие ценности — коллективизм, стремление к справедливости, ориентация на общинно-коллективистские (евразийские), а не на индивидуалистические (западноевропейские) ценности существования. Для белорусов, как и для русских и украинцев, основным является не личность, а коллектив, общество с идеалами братской любви и солидарности. Формирование духовных ценностей белорусского народа во многом связано с влиянием православно-византийского духовного наследия.

Для белорусов характерным является уважение права, законопослушание. Одно из главных мест в этой системе занимает толерантность (высокая степень национальной, расовой, конфессиональной и др. видов терпимости), трудолюбие, бережное отношение к земле и дому. Толерантность белорусов связана не только с поликонфессиональной средой, но и выступает как жизненная необходимость поддержания сложного равновесия, баланса разнонаправленных сил и влияний, в сфере которых постоянно оказывался белорусский народ на протяжении своей истории. Исключительная любовь к родной земле, привязанность к родным местам, хозяйственность, бережливость, трудолюбие, преданность семье и семейно-родовая солидарность — характерные черты белорусов.

Сегодня приоритетными для белорусов становятся такие ценности, как белорусская государственность, культура, язык, ценности собственной истории, национальных традиций, обычаев, общечеловеческих идеалов добра, правды, справедливости, соблюдения прав человека.

Отмечается неоднородность белорусской ментальности в зависимости от местонахождения: так для Гродненщины и других районов Западной Беларуси, развивающихся под влиянием католической Польши, Литвы и протестанской этики Западной Европы, характерна индивидуализация жизни; в Полесье же преобладает культ сельской общины; в белорусском Поозерье, граничащим с Россией, проявляется православная соборность.
В последнее время все чаще слышен призыв к единению славян, который зародился в глубине веков. Объединительная программа всех славян, т.е. восточных, южных и западных (европейских), католиков, православных и др., именуется панславизмом. Еще во времена Московской Руси, в царствование Алексея Михайловича, появились идеи и попытки такого объединения. Панславянские идеи поддерживал известный русский дипломат XVII столетия Афанасий Ордын-Нащекин, подчеркивая необходимость тесного союза с Польшей, призывая к объединению двух мощных государств и двух конфессий – католиков и православных. Позже, рассматривая различия исторических путей других славянских народов (болгар, сербов, чехов и др.), К. Леонтьев дополнил понятие "славизм" понятием "славянства", под которым понимал племенную совокупность славян. Сравнивая же исторический процесс в России и на Западе, он, как и Н.Данилевский, был убежден, что славянской России необходимы внутренняя сила, крепость организации и дисциплины, чтобы защитить свою независимость от европейского натиска. Дистанцируясь от Запада (испытывая "тяготение на почтительном расстоянии"), Россия в то же время должна в себя вобрать и некоторые западные тенденции.
О необходимости единения славян говорили и представители меньших славянских народов. Сторонниками единения славянства были, например, примыкавшие к протестанцизму мыслители Великого княжества Литовского С. Будный, Л. Зизаний, М. Смотрицкий, католики из этого же княжества Я. Веслицкий, Н. Гусовский, Я. Длугаш и др.

Ближе к нашему времени начинает укрепляться более целостное, не ограниченное конфессиональными рамками представление о путях единения славян. Влиятельными центрами утверждения единства славянства были в разные времена такие города, как Охрид, Тырново, Киев, Новогрудок, Белград, Минск. Но наибольшее влияние достигла Прага, которую называли "Славянскими Афинами". В ней и состоялся в 1848 году первый Славянский конгресс, сформировалась сильная школа ученых-славистов. Здесь, в Праге, проводились такие ответственные форумы, как Всеславянский съезд прогрессивных студентов (1908), I Международный съезд славянистов (1929) и, наконец, в 1998 году Всеславянский съезд.
На рубеже XX-XXI столетий одновременно с интеграцией западноевропейских стран происходит процесс дезинтеграции славянского мира, сопровождающейся разрушением СССР, расчленением СФРЮ, Чехословакии. Славянские духовные ценности, как проявление особой человеческой цивилизации и ментальности исторически уникальны и во многом самодостаточны.

В наши дни некоторого разъединения славян особенно актуально звучит призыв Н. Данилевского к мужеству, единодушию, твердой вере в величие славянских народов.

Как ответит на этот вызов времени восточнославянский мир, как сохранить и умножить славянские духовные ценности и единство славян на рубеже веков, каковы приоритеты дальнейшего развития человечества? Эти вопросы призывают белорусский народ и современные славянские народы к их величию, мужеству и твердой вере в свою историческую миссию единения различных народов, национальностей, конфессий, культур в контексте становления собственной государственности и глобализации мировой истории.