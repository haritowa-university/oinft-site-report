\subsection{Обзор технологии HTML}

\gls{html} -- стандартизированный язык разметки страниц во всемирной паутине. Большинство веб-страниц содержат описание разметки на языке \gls{html} (или XHTML). Язык \gls{html} интерпретируется браузерами; полученный в результате интерпретации форматированный текст отображается на экране монитора компьютера или мобильного устройства.

Язык HTML был разработан британским учёным Тимом Бернерсом-Ли приблизительно в 1986—1991 годах в стенах ЦЕРНа в Женеве в Швейцарии. \gls{html} создавался как язык для обмена научной и технической документацией, пригодный для использования людьми, не являющимися специалистами в области вёрстки. \gls{html} успешно справлялся с проблемой сложности SGML путём определения небольшого набора структурных и семантических элементов — дескрипторов. Дескрипторы также часто называют <<тегами>>. С помощью \gls{html} можно легко создать относительно простой, но красиво оформленный документ. Помимо упрощения структуры документа, в \gls{html} внесена поддержка гипертекста. Мультимедийные возможности были добавлены позже.

Изначально язык \gls{html} был задуман и создан как средство структурирования и форматирования документов без их привязки к средствам воспроизведения (отображения). В идеале, текст с разметкой \gls{html} должен был без стилистических и структурных искажений воспроизводиться на оборудовании с различной технической оснащённостью (цветной экран современного компьютера, монохромный экран органайзера, ограниченный по размерам экран мобильного телефона или устройства и программы голосового воспроизведения текстов). Однако современное применение \gls{html} очень далеко от его изначальной задачи. Например, тег \lstinline[language=HTML]{<table>} предназначен для создания в документах таблиц, но иногда используется и для оформления размещения элементов на странице. С течением времени основная идея платформонезависимости языка \gls{html} была принесена в жертву современным потребностям в мультимедийном и графическом оформлении.