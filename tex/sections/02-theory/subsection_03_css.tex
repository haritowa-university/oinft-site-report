\subsection{Обзор технологии CSS}

\gls{css} -- формальный язык описания внешнего вида документа, написанного с использованием языка разметки.

\gls{css} используется создателями веб-страниц для задания цветов, шрифтов, расположения отдельных блоков и других аспектов представления внешнего вида этих веб-страниц. Основной целью разработки \gls{css} являлось разделение описания логической структуры веб-страницы (которое производится с помощью \gls{html} или других языков разметки) от описания внешнего вида этой веб-страницы (которое теперь производится с помощью формального языка \gls{css}). Такое разделение может увеличить доступность документа, предоставить большую гибкость и возможность управления его представлением, а также уменьшить сложность и повторяемость в структурном содержимом. Кроме того, \gls{css} позволяет представлять один и тот же документ в различных стилях или методах вывода, таких как экранное представление, печатное представление, чтение голосом (специальным голосовым браузером или программой чтения с экрана), или при выводе устройствами, использующими шрифт Брайля\cite{wiki:CSS}.

До появления \gls{css} оформление веб-страниц осуществлялось исключительно средствами \gls{html}, непосредственно внутри содержимого документа. Однако с появлением \gls{css} стало возможным принципиальное разделение содержания и представления документа. За счёт этого нововведения стало возможным лёгкое применение единого стиля оформления для массы схожих документов, а также быстрое изменение этого оформления.

Преимущества:
\begin{itemize}
	\item Несколько дизайнов страницы для разных устройств просмотра. Например, на экране дизайн будет рассчитан на большую ширину, во время печати меню не будет выводиться, а на КПК и сотовом телефоне меню будет следовать за содержимым;
	\item Уменьшение времени загрузки страниц сайта за счет переноса правил представления данных в отдельный \gls{css}-файл. В этом случае браузер загружает только структуру документа и данные, хранимые на странице, а представление этих данных загружается браузером только один раз и может быть закэшировано;
	\item Простота последующего изменения дизайна. Не нужно править каждую страницу, а достаточно лишь изменить \gls{css}-файл;
	\item Дополнительные возможности оформления. Например, с помощью \gls{css}-вёрстки можно сделать блок текста, который остальной текст будет обтекать (например для меню) или сделать так, чтобы меню было всегда видно при прокрутке страницы.
\end{itemize}

Недостатки:
\begin{itemize}
	\item Различное отображение вёрстки в различных браузерах (особенно устаревших), которые по-разному интерпретируют одни и те же данные \gls{css};
	\item Часто встречающаяся необходимость на практике исправлять не только один \gls{css}-файл, но и теги HTML, которые сложным и ненаглядным способом связаны с селекторами \gls{css}, что иногда сводит на нет простоту применения единых файлов стилей и значительно увеличивает время редактирования и тестирования.
\end{itemize}