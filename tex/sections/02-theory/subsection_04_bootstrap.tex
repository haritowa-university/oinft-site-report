\subsection{Обзор фреймворка Bootstrap}\label{theory:bootstrap}

Для ускорения разработки и улучшения результата при работе над данным заданием, был использован фреймворк Bootstrap, созданный и поддерживаемый компанией Twitter.

\textit{Bootstrap} -- свободный набор инструментов для создания сайтов и веб-приложений. Включает в себя \gls{html}- и \gls{css}-шаблоны оформления для типографики, веб-форм, кнопок, меток, блоков навигации и прочих компонентов веб-интерфейса, включая \textit{JavaScript}-расширения, описанной в разделе \cite{wiki:bootstrap}. 

Основные инструменты Bootstrap:
\begin{itemize}
	\item \textbf{Сетки} — заранее заданные размеры колонок, которые можно сразу же использовать, например ширина колонки 140 px относится к классу .span2 (.col-md-2 в третьей версии фреймворка), который можно использовать в CSS-описании документа;
	\item \textbf{Шаблоны} — фиксированный или резиновый шаблон документа;
	\item \textbf{Типографика} — описания шрифтов, определение некоторых классов для шрифтов, таких как код, цитаты и т. п;
	\item \textbf{Медиа} — представляет некоторое управление изображениями и видео;
	\item \textbf{Таблицы} — средства оформления таблиц, вплоть до добавления функциональности сортировки;
	\item \textbf{Формы} — классы для оформления форм и некоторых событий, происходящих с ними;
	\item \textbf{Навигация} — классы оформления для панелей, вкладок, перехода по страницам, меню и панели инструментов;
	\item \textbf{Алерты} — оформление диалоговых окон, подсказок и всплывающих окон.
\end{itemize}

