\subsection{Обзор подходов к созданию веб страниц}

Условно разработку веб страниц можно разделить на две категории:
\begin{itemize}
	\item создание страниц при помощи конструкторов или аналогичных инструментов;
	\item создание страниц при помощи \gls{html}.
\end{itemize}

При создании страниц при помощи конструкторов, пользователь взаимодействует с \gls{wysiwyg} интерфейсом, который позволяет при помощи мыши создать страницу из заранее подготовленных настраиваемых элементов и блоков. Данный подход не требует наличия технических навыков и доступен каждому, однако часто ведёт к худшему результату, меньшей гибкости и платной модели использования.

Вторая категория подразумевает использование стандартного инструментария веб-разработчика:

\begin{itemize}
	\item \gls{html};
	\item \gls{css};
	\item \gls{js}.
\end{itemize}

В рамках выполнения данного задания, был выбран второй способ создания сайта с поддержкой библиотеки Bootstrap\ref{theory:bootstrap}.