\sectioncentered*{Заключение}
\addcontentsline{toc}{section}{Заключение}

Современная Беларусь представляет собой переходное общество как в политическом, так и в общецивилизационном понимании. Находясь на стыке двух цивилизаций (западной и восточной), Беларусь испытывает на себе влияние каждой из них. В данной связи встает вопрос о принадлежности белорусов к определенной культурной, географической и духовной общности, обусловливающей их специфику, отличие от других народов.

Среди цивилизационных концепций исторического развития интерес представляют теории Н.Я. Данилевского, О. Шпенглера, А. Тойнби, С. Хантингтона С позиций Н. Данилевского развитие цивилизаций имеет циклический характер, постепенно проходя стадию зарождения, когда происходит процесс формирования "самобытной цивилизации" на основе принадлежности к одной языковой группе; период развития, в ходе которого оформляется культурная и политическая независимость.

Подобно Н. Данилевскому, О. Шпенглер также рассматривает исторический процесс как рождение, рост и разрушение культурных типов. В книге "Закат Европы" (1918-1922) он приходит к выводу о гибели западной культуры, находящейся на цивилизационной стадии развития. С его позиций, цивилизация – это кульминационная стадия развития любой культуры, ее распад, признаки которого проявляются в росте научно-технического прогресса, обезличивании человека, снижении творческой активности. В теории А. Тойнби отвергается тезис о замкнутости культурных типов и непроницаемости цивилизаций, а в качестве связующего фактора он предложил создание универсальной церкви. В основе классификации С. Хантингтона также положен религиозный признак, в соответствии с которым он разделяет все существующие цивилизации на англо-саксонскую, католическо-протестанскую, православно-славянскую, мусульманскую, конфуцианскую, синтоистскую, латиноамериканскую и африканскую. Беларусь в его концепции оказывается на границе католическо-протестанской и православно-славянской цивилизации.

Национальное самосознание формируется на базе мировоззренческих основ, принадлежности к определенному цивилизационному типу. Беларусь, находясь на стыке цивилизаций, впитала в себя особенности как православно-славянской, так и католическо-протестанской цивилизаций, что привело к сочетанию в белорусском менталитете восточного мистицизма и западного рационализма. Национальное самосознание проявляется в самоидентификации с устойчивой, исторически сложившейся общностью (нацией), основанной на единстве языка, культуры, традиций, а также ощущении своей уникальности, самобытности, отличия от других народов: "мы – белорусы".
