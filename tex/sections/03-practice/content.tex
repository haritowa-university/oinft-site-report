\section{Основные этапы, проблемы и представители философской мысли Беларуси}
\label{sec:practice}


Формирование условий, подготовивших возможность возникновения философии Беларуси, следует связать с принятием христианства во времена Киевской Руси. Начало процессу христианского просвещения было положено полоцкой княжной Рогнедой, принявшей христианство и постригшейся в монахини под именем Анастасии. Своей христианско-просветительской деятельностью широко известна также полоцкая княжна Предслава, принявшая монашество под именем Ефросиньи (Полоцкой). Становление же профессиональной философской деятельности в Беларуси произошло в период Возрождения; оно связано с именем белорусского первопечатника и гуманистаФранциска Скорины.

Развитие белорусской философии осуществлялось в контексте эволюции всей европейской культуры. Поэтому не удивительно, что белорусская философская мысль эпохи Возрождения воплощала в себе основные черты философии европейского Ренессанса. К таковым следует отнести:

1) антропоцентризм, т.е. идею самодостаточности человеческого бытия, когда в центре внимания философов выступала проблема исторического предназначения человека;

2) идею абсолютной духовной свободы, сочетавшей в себе концепцию гносеологического оптимизма с представлением о безграничных возможностях человека в деле практического преобразования природы, общества на началах разума;

3) натурализм, выступавший в качестве основополагающего принципа интеграции мироздания и самого человека.

Вместе с тем своеобразие социально-экономического и общественно-политического развития Беларуси отложило свой отпечаток на развитие национальной духовной традиции, обусловило специфические черты эволюции философской мысли. Следует отметить, например, что в трактовке индивидуальной свободы, исторического предназначения человека здесь выпукло представлена идея ограничения абсолютной свободы интересами всеобщего блага. Белорусский гуманизм не абсолютизировал свободу, но трансформировал ее понимание в идею социального служения. Она конкретизировалась демократической, просветительской интенцией белорусского Ренессанса, стремлением сделать культурно-исторические ценности достоянием всего общества, всего народа. Этим объясняются особенности стиля изложения, практикуемого белорусскими мыслителями, их стремление к ясному, простому, доходчивому языку. Для белорусской культуры была характерна задача актуализации христианско-гуманистических ценностей, творческого синтеза идей натурализма и теологизма (Франциск Скорина, Сымон Будный). Белорусский ренессансный гуманизм развивался в тесной связи с Реформацией, широким социальным движением, захватившем самые широкие слои общества. В отличие от западноевропейского он преодолел узкие рамки духовных, художественных элит.

Реформация в Великом княжестве Литовском оказала существенное влияние на динамику культурных процессов, активизировала общественно-политическую деятельность практически всех социальных слоев. Она стимулировала проведение социально-экономических реформ, развитие образования. Особое значение приобрело радикально-реформационное движение, инициировавшее обсуждение целого ряда философских проблем: природа духовной свободы, роль разума в познавательской деятельности, соотношение веры и разума, отношение к античному духовному наследию. Возникший в культурном контексте Реформации протестантизм существенно снизил статус церкви как посредника в общении человека с Богом, лишил ее монополии на толкование Священного Писания, вывел индивида на непосредственный контакт с Абсолютом. Тем самым был открыт путь к развитию свободной мысли, выходившей из-под опеки церковного авторитета, развитию морально-этического, правового творчества. Что касается белорусской духовности, то здесь отчетливо проявило себя ее рационалистическое направление (С. Будный). По сути говоря, было положено начало научному изучению библейских текстов, а это обстоятельство имело далеко идущие социально-культурные последствия: научно-рациональный подход должен был необходимо распространиться на изучение природы и общества. Протестантская духовность подчеркивала ценность посюсторонней жизни, способствовала воспитанию предпринимательской активности, моральной ответственности личности (Ф. Скорина, М. Литвин). Реформация и гуманизм способствовали утверждению личных прав: права на неприкосновенность личности, права на собственность, права на общественно-политическую активность. Получила распространение политическая философия, отражающая в себе основные черты юридического мировоззрения. Вспомним, например, разрабатывавшиеся в реформационный период Статуты Великого княжества Литовского 1529, 1566 и 1588 гг.

Философская и общественно-политическая мысль белорусского Возрождения включает в себя следующие этапы:

1. Философия раннего Возрождения, начала ХVI–50-60-е гг. ХVI в. В социально-экономическом плане это был период интенсификации ремесленного производства, торговли, роста и возвышения городов, усиления межклассовых и межсословных противоречий, начала формирования гражданского и национального самосознания, осознания необходимости перемен в экономической и политической жизни. Главная духовная интенция этой эпохи – идея рационального переустройства общества.

2. Философия Возрождения и Реформации, вторая половина ХVI в. Этот период включил в себя широкое социальное движение, попытки практического социального реформирования (судебная и административная реформы, борьба за городское самоуправление, Статуты ВКЛ 1566 и 1588 гг., создание Главного Литовского трибунала, возникновение лютеранства, кальвинизма, антитринитаризма). Философия Беларуси пыталась пересмотреть традиционные теологические, социально-политические представления, шел процесс формирования буржуазного мировоззрения, обоснования идей рационализма, политического реформизма, естественногоправа (М. Литвин, С. Будный, А. Волан).

