\section{Значение идей белорусского Просвещения в реализации программы национального Возрождения}
\label{sec:practice}

Философия позднего Возрождения, контрреформации и барокко (конец ХVI–ХVII вв.). Содержание этого периода может быть обозначено такими признаками, как: развертывание контрреформации, ослабление и закат реформационно-гуманистического движения, усиление тенденций к морализированию. Формулируются задачи защиты национально-культурного своеобразия белорусов, их национально-политического самоопределения. Это было время утверждения схоластического типа философствования. Гуманистические тенденции этого периода, представленные Казимиром Лыщинским, отчасти Симеоном Полоцким, отступают перед философскими и художественно-эстетическими идеалами барокко и схоластики.

В истории философской мысли Беларуси эпохи Ренессанса выделяют следующие направления:

а) радикальное реформационно-гуманистическое (Якуб из Калиновки, Мартин Чаховиц, Стефан Зизаний). Для него было характерно критическое отношение к предшествующей традиции (философии античности, схоластике, церковному учению). Внимание мыслителей концентрировалось на сущности личной веры и Божественном откровении, явленном в Священном писании. Схоластическое богословие только замутняет свободный разум человека. По сути, это была своеобразная попытка тотального разрыва с традиционной культурой, ее ценностями. Хотя, справедливости ради, нужно отметить интерес этого направления к таким ценностям, совсем не чуждым христианству, как равенство, братство, человеколюбие. Представители этого направления пытались осмыслить эгалитарно-демократические идеи Евангелия;

б) умеренное реформационно-гуманистическое (Ф. Скорина, Н. Гусовский, С. Будный, А. Волан, Л. Зизаний, С. Полоцкий). Оно стремилось к выработке компромиссных форм соединения, согласования античной, философской традиции с ценностями средневековой христианской и гуманистически-ренессансной культур. Мыслители этого направления были уверены в возможности использования в условиях новой культуры духовных форм прошлого. Они активно осваивали платонизм, аристотелизм, этические и политические учения стоицизма, с уважением относились к патристике (Тертуллиан, Августин, Григорий Нисский, Псевдо-Дионисий Ареопагит), но отвергали схоластику, тем более что последняя выступила в качестве идейного орудия контрреформации. Порок схоластики гуманисты усматривали в ее излишней академичности, оторванности от повседневной жизни, общественных процессов. Умеренные гуманисты-реформаторы видели нетленные образцы общественного переустройства в античных философских концепциях, реальном практическом опыте европейских народов, в библейских текстах. Они обогатили отечественную философскую мысль новыми идеями и проблемами. Их относительная слабость заключалась, видимо, в недооценке самой философии, точнее, недооценке самодостаточности философского знания. Ценность философии они усматривали в ее возможном практическом приложении. По их мнению, философское знание должно было выступить в качестве орудия социальных преобразований. Мировоззренческое значение философии, значимость профессионально-философской активности не учитывались в достаточной мере; философский беларусь национальный просвещение

в) атеистическо-гуманистическое; оно, порвав со схоластикой и теологией, обратилось к материалистическим идеям античности и Возрождения (С.Г. Лован, К. Бекеш, К. Лыщинский). Эти мыслители заимствовали и осмысливали античные решения проблемы происхождения мира; античные обоснования естественноприродной сущности человеческой морали; они отвергали идею Божественного откровения, концепцию врожденных идей; высоко ценили индивидуальный разум и опыт в качестве средств познания.

Оценивая гуманистически-ренессансную философскую мысль Беларуси, следует отметить, что она подготовила предпосылки для формирования нового типа мировоззрения.

Европейская философия того времени занималась обоснованием ценностей науки, научного познания, идеалов техногенного развития цивилизации (Ф. Бэкон, Р. Декарт, Б. Спиноза). Однако эти предпосылки не смогли реализоваться. Наступил период контрреформации. Схоласты безжалостно подавляли альтернативные направления. Из Речи Посполитой были изгнаны социниане, иезуиты учинили расправу над К. Лыщинским. Господство схоластики было установлено с помощью "вненаучных" средств: идейное давление, поддержка со стороны власть предержащих; изгнание инакомыслящих. Однако схоластика как способ философствования, оставшийся применительно к другим европейским странам в далеком прошлом, сохранила известную связь с идеями ренессансного гуманизма, научной философии и естествознания. Тем самым она перерождалась, открывая пути новым философским идеям, не совместимым с ее собственными мировоззренческими основаниями.

Развитие духовной культуры Беларуси ХVII–нач. ХVIII вв. проходило под знаком идейной борьбы православия и униатства. Философская мысль этого времени концентрировалась в сфере влияния католических орденов (иезуитов, доминиканцев). Формировалась поздняя, "виленская" схоластика, сосредоточившаяся на разработке этических, эстетических, педагогических идей. Развертывались дискуссии между представителями различных конфессий: православной (М. Ващенко, Л. Карпович, М. Смотрицкий), униатской (Л. Кревза, И. Кунцевич), католической (Я. Альшевский, А. Баболь, Я.А. Кулеш). В конце ХVIII–начале ХIХ в. происходило становление на территории Беларуси и Литвы классического естествознания, что требовало философского осмысления его оснований (М. Почебут, Ю. Мицкевич, Я. Снядецкий).

В первой половине ХIХ в. наблюдается закат традиций Просвещения, смена духовно-ценностных ориентаций в связи с вхождением Беларуси в состав Российской империи. Духовный проект филоматов ("стремящихся к знанию"), выдвинутый в Виленском университете, задал ориентацию на идеи национально-культурного и государственного возрождения. Однако в целом собственно философская традиция в Беларуси прерывается.

Общественно-политическая мысль полнилась идеями поиска национальной идентификации.

Такого рода эволюция была инициирована этнографическими исследованиями, проведенными в Беларуси. С середины ХIХ в. осмысление этого круга идей осуществлялось в художественной литературе (А. Мицкевич, Я. Чечот, У. Сырокомля, Я. Барщевский, В. Дунин-Мартинкевич, Я. Лучина, Ф. Богушевич). В ХХ в. эта традиция была продолжена Я. Купалой, Я. Коласом, М. Богдановичем, М. Горецким. В конце 20-х гг. заканчивается очередной период в развитии философской мысли Беларуси, ее эволюция протекала далее в контексте советской философии.

Распад Советского Союза и образование суверенной Республики Беларусь дали новые импульсы поискам национальной идентификации белорусов, стимулировали обращение к национальной духовной традиции, освоению ценностей, выработанных в белорусской философской культуре

Отличительной чертой белорусской философии является чуткое и незамедлительное реагирование на ключевые события в жизни народа и отражение культурно-политической истории Беларуси в национальном общественно-политическом творчестве.

Главным фактором, интегрирующим разнообразные философско-социологические исследования различных этапов отечественной общественной мысли является изучение проблемы человека с позиций гуманизма.



3. Значение идей белорусского Просвещения в реализации программы национального Возрождения и становлении национальной идеи как консолидирующей духовной силы современного белорусского общества (Ф. Скорина, С. Будный, А. Волан и др.)



Зарождение общественно-политической, эстетической и философской мысли на территории Беларуси относится к X—XI вв. и связано с созданием государственности и развитием феодальных отношений. Первые элементы философской культуры находят свое отражение в притчах и обрядовых гимнах, сказках и былинах. Большое значение для развития философской мысли имело распространение письменности на территории Беларуси, что связано с принятием христианства.

В X—XI вв. появляются первые литературные памятники, так называемые "фларигелии" — сборники фрагментов, которые включали мысли античных философов, идеи античного гуманизма, вопросы этики, эстетики, общей психологии. Сохранилось 29 рукописных книг, отдельных грамот и надписей. Среди них такие памятники культуры, как "Супрасльская летопись", "Туровское Евангелие", "Диоптра" и др. В частности, "Диоптра" включала все основные христианские идеи о человеке, его месте в мире и основных ценностях. В тексте есть много цитат из Библии, из произведений восточнохристианских отцов церкви. В "Диоптре" утверждается идея изменения Вселенной, все связано со временем. То, что прошло, уже не существует, то, что есть, — мгновенно, а будущее непредсказуемо и неизвестно. На территории Беларуси распространились произведения, общие для русской, белорусской и украинской культур. Среди них — "Повесть временных лет", "Слово о полку Игореве", сочинения митрополита Иллариона, "Поучение" Владимира Мономаха и др.

Развитие общественно-политической мысли в Беларуси в XI—ХIII вв. сопровождалось распространением просветительских идей. Возникло движение "нехтивцев", представители которого посвящали всю свою деятельность и жизнь просветительству общества (К. Смолятич, К. Туровский, Е. Полоцкая). Просветительские идеи отражены в таких памятниках XII в., как "Полоцкое Евангелие", "Послание Климента Смолятича", "Сочинения" К. Туровского, "Сказание о Христе и Антихристе" и др. Христианская идеология принесла на белорусскую землю не только образованность, письменную культуру, но и жестокие нравы. Распространение христианства на славянских территориях происходило в борьбе с еретиками, характеризовало собой противоположность мировоззрений христиан и язычников.

Особенное развитие получили ереси в ХIV—ХV вв. Движение еретиков связано с критикой идеологии церкви, выступлениями против духовного угнетения, утверждением принципа веротерпимости и ценности человека. Для философской мысли XVI в. характерно развитие вольнодумства, свободомыслия и идей гуманизма. На развитие философской мысли большое влияние оказывали общекультурные явления в Европе, такие как Возрождение и Реформация. В общественном мировоззрении начался процесс формирования новых ориентации, связанных с утверждением антирелигиозных ценностей. Развитие светской культуры встретило жестокий отпор со стороны церковных властей. В частности, результатом борьбы католической церкви против реформаторских идей явилась Брестская уния (1595), которая узаконила и усилила католическую реакцию.

Видный представитель эпохи вольнодумства — Франциск Скорина (ок. 1490—1541 гг.). Его жизнь и деятельность, интересы и ориентации отвечали атмосфере эпохи Возрождения. Получив блестящее образование, Ф. Скорина основал типографию, переводил, издавал и распространял литературу. Его заслугой является первое светское издание Библии (в то время право издания Библии монопольно принадлежало церкви, даже чтение ее было запрещено простым людям, и только представители духовенства имели право на чтение текстов священной книги, их интерпретацию). Издание Библии, да еще с предисловием, где Скорина говорил об историческом и культурном, а не только о религиозном значении Библии, было прямым вызовом церковным порядкам. Философ проповедовал идеи веротерпимости, защищал принципы личных отношений с Богом (без посредничества церкви), отстаивал полезность образования и идеи, которая утверждает достоинство человека. Заслугой Скорины является то, что он печатал книги и излагал свои взгляды на белорусском языке, тем самым, содействуя развитию национальной культуры.

Критическое отношение к церковной политике и Священному писанию высказывал видный просветитель XVI в. Симон Будный (ок. 1530—1593 гг.). Мыслитель-гуманист сделал перевод Библии, поставив перед собой задачу исправить "глупости переписчиков" и "хитрость еретиков", как "ошибки прежних переводчиков". Его интерпретация библейских текстов содержит первые приметы атеистического мировоззрения.

Дальнейшее развитие идей вольнодумства, атеизма связано с деятельностью Казимира Лыщинского (1634—1689 гг.) в мрачную эпоху господства католической церкви в Речи Посполитой. Лыщинский казнен за распространение атеистических мыслей. Его рукопись "О несуществовании Бога", которая послужила основой для доносов, а потом и наказания, была сожжена.

В ХVI—ХVIII вв. в Беларуси усилилась свободолюбивая тенденция в развитии общественно-политической мысли, получили распространение идеи, которые развивали С. Щадурский, К. Нарбут, Б. Добшевич и др. Они протестовали против церковного угнетения, схоластики, отстаивали права человека на свободу мысли и действий. К концу XVIII в. Беларусь вошла в состав России, что дало новые возможности для развития духовной жизни под воздействием русской культуры. Постепенно расширялось движение против церковного засилья и власти помещиков. Оно характеризовалось рядом крестьянских бунтов, среди которых особо выделяется крестьянское восстание во главе с Кастусем Калиновским (1838—1864 гг.). Идеи революционных демократов нашли свое отражение и в литературе того времени (Ф. Богушевич, Я. Лучина, М. Богданович).

В 80-е гг. XIX в. на территории Беларуси начинают распространяться идеи марксизма, главный принцип которых — обоснование революционного преобразования общества. В разное время здесь работали известные марксисты И.В. Бабушкин (1900—1917 гг.), А.Ф. Мясников (1917—1919 гг.).

После революции 1917 г. центрами развития философской мысли становятся Белгосуниверситет, Коммунистический университет, Инбелкульт (с 1929 г. Академия наук). Там работали известные профессора Б.Э. Быковский, С.Я. Вольфсон, С.З. Каценбоген. В 1923 г. было организовано марксистское товарищество, а в 1927 г.— товарищество историков-марксистов. В 1921 г. издан первый учебник по диалектическому материализму С.Я. Вольфсона (выдержал семь изданий).

В философии советского времени можно выделить отдельные периоды, основой для которых служит развитие тех или иных проблем и вопросов. Так, 20—30-е гг. характеризуются интересом к анализу отношений философии и собственно наук, разработками проблем категориального аппарата и методов философской науки. Белорусские философы участвовали в дискуссии против формализма и механицизма.

30-е гг.— период активизации движения так называемых националистов, которые выступили против социалистической идеи и политики, рассматривали белорусскую нацию как особенную, что вело к самоизоляции. Но эти идеи не получили распространения.

Основная задача философской науки в советское время постепенно сводилась к популяризации идей марксизма-ленинизма, прославлению успехов на пути строительства социалистического общества, пропаганде идей о процветании наций и культур, прогрессивных изменений в социальной сфере. Одновременно критиковалось буржуазное общество, его идеология и наука. В философских разработках, посвященных проблемам социального развития, преобладала абсолютизация классового подхода, научная аргументация часто заменялась навешиванием политических ярлыков. В развитии общественно-политической мысли сталинского периода выявилась чрезмерная идеологизация философской науки, что привело к снижению ее познавательных функций, в значительной степени ослабило эвристические возможности философского мышления.

В 50—80-х гг. белорусской философской мыслью разрабатывались проблемы диалектики и ее законов, вопросы логики и методологии научного познания и др. Значительное место в философии этого периода занимают вопросы истории философии, в том числе и истории философской мысли Беларуси. Белорусские ученые издали ряд трудов, посвященных таким мыслителям, как С. Будный, Я. Белобоцкий, Ю. Доманевский, М. Смотрицкий, Л. Филипович, К. Нарбут, К. Калиновский и др.

В последние десятилетия значительный вклад в развитие философской мысли внесли белорусские философы К.П. Буслов, В.И. Степанов, И.Н. Лущицкий, Д.И. Широканов, Е.М. Бабосов, А.С. Майхрович, С.А. Падокшин, B.C. Степин, А.В. Бодаков, В.М. Конон и др. В истории развития философской мысли Беларуси отражались этапы ее общественного развития, а сама философия выступала активным фактором социального и духовного развития белорусского народа, его культуры.

