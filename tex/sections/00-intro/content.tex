\sectioncentered*{Введение}
\addcontentsline{toc}{section}{Введение}

Национальное самосознание - осознание людьми своей принадлежности к определенной социально-этнической общности и ее положения в системе общественных отношений. Национальное самосознание проявляется в идеях, взглядах, мнениях, чувствах, эмоциях, настроениях и выражает содержание, уровень и особенности представлений членов нации о своей определенной идентичности и отличии от представителей других общностей; о национальных ценностях и интересах; истории нации, ее нынешнем состоянии и перспективах развития; месте своей социально-этнической общности во внутригосударственных, межгосударственных и межнациональных отношениях.

Отличают как рациональные компоненты национального самосознания (собственно осознание своей принадлежности к нации), так и эмоциональные (напр. сопереживание своего единства с другими членами социально-этнической общности). Национальное самосознание является важнейшей составной частью, основой национального сознания. Национальное самосознание существует не только на уровне индивида, но и над личностью, в том числе в объективированных массовых формах общественного сознания: в языке, в произведениях народного творчества и профессионального искусства, научной литературе, нормах морали и права и т.д.

Интенсивность проявления национального самосознания у отдельных представителей этнической общности далеко не одинакова. Частично или полностью им не обладают дети. У взрослых членов этноса, как правило, оно ослаблено в тех случаях, когда они не имеют контактов с представителями других этнических общностей. В таком положении чаще всего оказываются сельские жители, у которых может преобладать локальное или региональное самосознание.
Национальное самосознание проявляется и в отношениях к другим нациям. Там, где между двумя этническими общностями складывались отношения сотрудничества, вырабатывалась в основном положительная установка в отношениях друг к другу, предполагающая терпимость к существующим национальным различиям. Если отношения между народами были далекими, не затрагивающими их жизненных интересов, эти народы могли быть не настроены враждебно друг к другу, но могли и не питать друг к другу особой симпатии. Иное дело, когда народы длительное время находились в состоянии конфликта и вражды. Тогда вырабатывалась в основном враждебная психологическая установка. При этом разница в образовании респондентов, имущественное, классовое положение не играют особой роли. Все эти предубеждения предопределены историческими, психологическими факторами и прочно вошли в сознание и установки людей.